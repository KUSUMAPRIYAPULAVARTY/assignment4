\documentclass[journal,12pt,twocolumn]{IEEEtran}

\usepackage{setspace}
\usepackage{gensymb}

\singlespacing


\usepackage[cmex10]{amsmath}

\usepackage{amsthm}

\usepackage{mathrsfs}
\usepackage{txfonts}
\usepackage{stfloats}
\usepackage{bm}
\usepackage{cite}
\usepackage{cases}
\usepackage{subfig}

\usepackage{longtable}
\usepackage{multirow}

\usepackage{enumitem}
\usepackage{mathtools}
\usepackage{steinmetz}
\usepackage{tikz}
\usepackage{circuitikz}
\usepackage{verbatim}
\usepackage{tfrupee}
\usepackage[breaklinks=true]{hyperref}
\usepackage{graphicx}
\usepackage{tkz-euclide}

\usetikzlibrary{calc,math}
\usepackage{listings}
    \usepackage{color}                                            %%
    \usepackage{array}                                            %%
    \usepackage{longtable}                                        %%
    \usepackage{calc}                                             %%
    \usepackage{multirow}                                         %%
    \usepackage{hhline}                                           %%
    \usepackage{ifthen}                                           %%
    \usepackage{lscape}     
\usepackage{multicol}
\usepackage{chngcntr}

\DeclareMathOperator*{\Res}{Res}

\renewcommand\thesection{\arabic{section}}
\renewcommand\thesubsection{\thesection.\arabic{subsection}}
\renewcommand\thesubsubsection{\thesubsection.\arabic{subsubsection}}

\renewcommand\thesectiondis{\arabic{section}}
\renewcommand\thesubsectiondis{\thesectiondis.\arabic{subsection}}
\renewcommand\thesubsubsectiondis{\thesubsectiondis.\arabic{subsubsection}}


\hyphenation{op-tical net-works semi-conduc-tor}
\def\inputGnumericTable{}                                 %%

\lstset{
%language=C,
frame=single, 
breaklines=true,
columns=fullflexible
}
\begin{document}


\newtheorem{theorem}{Theorem}[section]
\newtheorem{problem}{Problem}
\newtheorem{proposition}{Proposition}[section]
\newtheorem{lemma}{Lemma}[section]
\newtheorem{corollary}[theorem]{Corollary}
\newtheorem{example}{Example}[section]
\newtheorem{definition}[problem]{Definition}

\newcommand{\BEQA}{\begin{eqnarray}}
\newcommand{\EEQA}{\end{eqnarray}}
\newcommand{\define}{\stackrel{\triangle}{=}}
\bibliographystyle{IEEEtran}

\providecommand{\mbf}{\mathbf}
\providecommand{\pr}[1]{\ensuremath{\Pr\left(#1\right)}}
\providecommand{\qfunc}[1]{\ensuremath{Q\left(#1\right)}}
\providecommand{\sbrak}[1]{\ensuremath{{}\left[#1\right]}}
\providecommand{\lsbrak}[1]{\ensuremath{{}\left[#1\right.}}
\providecommand{\rsbrak}[1]{\ensuremath{{}\left.#1\right]}}
\providecommand{\brak}[1]{\ensuremath{\left(#1\right)}}
\providecommand{\lbrak}[1]{\ensuremath{\left(#1\right.}}
\providecommand{\rbrak}[1]{\ensuremath{\left.#1\right)}}
\providecommand{\cbrak}[1]{\ensuremath{\left\{#1\right\}}}
\providecommand{\lcbrak}[1]{\ensuremath{\left\{#1\right.}}
\providecommand{\rcbrak}[1]{\ensuremath{\left.#1\right\}}}
\theoremstyle{remark}
\newtheorem{rem}{Remark}
\newcommand{\sgn}{\mathop{\mathrm{sgn}}}
\providecommand{\abs}[1]{\left\vert#1\right\vert}
\providecommand{\res}[1]{\Res\displaylimits_{#1}} 
\providecommand{\norm}[1]{\left\lVert#1\right\rVert}
%\providecommand{\norm}[1]{\lVert#1\rVert}
\providecommand{\mtx}[1]{\mathbf{#1}}
\providecommand{\mean}[1]{E\left[ #1 \right]}
\providecommand{\fourier}{\overset{\mathcal{F}}{ \rightleftharpoons}}
%\providecommand{\hilbert}{\overset{\mathcal{H}}{ \rightleftharpoons}}
\providecommand{\system}{\overset{\mathcal{H}}{ \longleftrightarrow}}
	%\newcommand{\solution}[2]{\textbf{Solution:}{#1}}
\newcommand{\solution}{\noindent \textbf{Solution: }}
\newcommand{\cosec}{\,\text{cosec}\,}
\providecommand{\dec}[2]{\ensuremath{\overset{#1}{\underset{#2}{\gtrless}}}}
\newcommand{\myvec}[1]{\ensuremath{\begin{pmatrix}#1\end{pmatrix}}}
\newcommand{\mydet}[1]{\ensuremath{\begin{vmatrix}#1\end{vmatrix}}}

\numberwithin{equation}{subsection}

\makeatletter
\@addtoreset{figure}{problem}
\makeatother
\let\StandardTheFigure\thefigure
\let\vec\mathbf

\renewcommand{\thefigure}{\theproblem}

\def\putbox#1#2#3{\makebox[0in][l]{\makebox[#1][l]{}\raisebox{\baselineskip}[0in][0in]{\raisebox{#2}[0in][0in]{#3}}}}
     \def\rightbox#1{\makebox[0in][r]{#1}}
     \def\centbox#1{\makebox[0in]{#1}}
     \def\topbox#1{\raisebox{-\baselineskip}[0in][0in]{#1}}
     \def\midbox#1{\raisebox{-0.5\baselineskip}[0in][0in]{#1}}
\vspace{3cm}
\title{Assignment 4}
\author{KUSUMA PRIYA\\EE20MTECH11007}

\maketitle
\newpage

\bigskip
\renewcommand{\thefigure}{\theenumi}
\renewcommand{\thetable}{\theenumi}
Download all python codes from 
\begin{lstlisting}
https://github.com/KUSUMAPRIYAPULAVARTY/assignment4/tree/master/codes
\end{lstlisting}
%
and latex-tikz codes from 
%
\begin{lstlisting}
https://github.com/KUSUMAPRIYAPULAVARTY/assignment4
\end{lstlisting}
%
 
 \section{QUESTION}
Prove that the following equations represent two straight lines.Also find their point of intersection and the angle between them
\begin{align}
 3y^2-8xy-3x^2-29x+3y-18=0   
\end{align}

%

\section{Explanation}
The general form of equation representing a pair of straight lines is 
\begin{align}
    ax^2+2hxy+by^2+2gx+2fy+c=0
\end{align}
These represent a pair of lines if the discriminant of the equation is zero, that is,
\begin{align}
\mydet{a&h&g\\h&b&f\\g&f&c}=0
\end{align}
\section{Solution}
The discriminant of (1.0.1) becomes
\begin{align}
    \mydet{-3&-4&-\frac{29}{2}\\-4&3&\frac{3}{2}\\-\frac{29}{2}&\frac{3}{2}&-18}
\end{align}
Expanding equation (3.0.1), we have discriminant equal to zero.\\
Hence given equation represents a pair of straight lines.
\section{Finding the individual lines}
Rewriting equation (1.0.1) and solving for $y$,
\begin{align}
y^2-y\left(\frac{8x-3}{3}\right)=\frac{3x^2+29x+18}{3}\\
\implies y^2-y\left(\frac{8x-3}{3}\right)+\left(\frac{8x-3}{6}\right)^2=\\\frac{3x^2+29x+18}{3}+\left(\frac{8x-3}{6}\right)^2\\
\left[y-\left(\frac{8x-3}{6}\right)\right]^2=\left[\frac{5}{6}(2x+3)\right]^2\\
\implies y=\frac{8x-3}{6}-\frac{5}{6}(2x+3),\\
y=\frac{8x-3}{6}+\frac{5}{6}(2x+3)
\end{align}
The individual line equations therefore become
\begin{align}
    x+3y+9=0\\-3x+y-2=0
\end{align}
\section{Intersection Point}
The augmented matrix for the set of equations represented in (4.0.7), (4.0.8) is
\begin{align}
\myvec{1&3&-9\\-3&1&2}
\end{align}
Row reducing the matrix
\begin{align}
 \myvec{1&3&-9\\-3&1&2}\xleftrightarrow[]{R_2\leftarrow R_2+3\times R_1}\myvec{1&3&-9\\0&10&-25}\\
 \xleftrightarrow[]{R_1\leftarrow R_1-\frac{3}{10}\times R_2}\myvec{1&0&-\frac{3}{2}\\0&10&-25}\\
 \xleftrightarrow[]{R_2\leftarrow \frac{R_2}{10}}\myvec{1&0&-\frac{3}{2}\\0&1&-\frac{5}{2}}\\
\text{Hence, the intersection point is}
\myvec{-\frac{3}{2}\\-\frac{5}{2}}
\end{align}
\begin{figure}[!ht]
\centering
\includegraphics[width=\columnwidth]{hw4plot.png}
\caption{plot showing intersection of lines}
\label{Fig}
\end{figure}
\section{Angle Between The Lines}
The direction vectors of lines in (4.0.7), (4.0.8) are
\begin{align}
    \vec{m_1}=\myvec{1\\-\frac{1}{3}}\\
    \vec{m_2}=\myvec{1\\3}
\end{align}
Their corresponding normal vectors are
\begin{align}
    \vec{n_1}=\myvec{\frac{1}{3}\\1}\\
    \vec{n_2}=\myvec{-3\\1}
\end{align}
Angle between two lines $\theta$ can be given by
\begin{align}
\cos \theta = \frac{\vec{n_1}^T\vec{n_2}}{\norm{\vec{n_1}}\norm{\vec{n_2}}}\\
=\frac{\myvec{\frac{1}{3}&1}\myvec{-3\\1}}{\sqrt{\left(\frac{1}{3}\right)^2 +1} \times \sqrt{(-3)^2 +1}}=0\\
\implies \theta = 90\degree
\end{align}
\end{document}

